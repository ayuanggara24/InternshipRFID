\section{Introduction}
\label{Introduction}
At present, Many industries today are using Radio Frequency Identification (RFID) technology \cite{Zaman2017}, RFID applies in applications such as attendance logging applications \cite{Eridani2015}, warehouse management \cite{xie2014efficiently}, library \cite{liu2017rf}, object tracking \cite{bu2015deterministic}, and others. RFID can create objects of "talking" technology, so RFID technology in key technologies from perceptual perception layer positions is very prominent \cite{pane2018qualitative}. RFID has several advantages over the traditional identification technologies. RFID does not require random tracks for communication and RFID tags can be read more \cite{Arjona2015}. RFID is relatively fast, and many tags can read simultaneously. The RFID system consists of RFID tags, RFID reader and PC \cite{Almaaitah2014}. Each of RFID tag has a unique ID that corresponds to some useful information (e.g. product, tracking, and position information), from unique ID and position information, users can quickly identify RFID tag locations \cite{Xu2016}.\par
While RFID widely use, it is important to note that RFID has weaknesses. The weakness in RFID systems is that it is possible to clone identification data \cite{basilio2016multifactor}. Cloning attacks make the application unsafe because it duplicates the original tags so that it threatens RFID applications that use tag authenticity to validate objects \cite{maleki2017new}. This cloning attack can result in financial loss to users \cite{shao2015protar}, \cite{huang2017dtd}.\par
Therefore, to improve system security and minimise duplication in RFID system in this research will apply Multi-Factor Authentication (MFA) method. MFA is a way to authenticate users by using multiple layers of the Authentication program. This is a secure means of authentication that can effectively prevent identity theft \cite{Venukumar2016}. In addition, using this method will be slavish in tracking the precision of the calculation of the tracking results \cite{Awangga2017}. The factor in this report is using One-Time Password (OTP). Thus, the existence of a system with some authentication steps can minimise the fraud and duplication that will occur \cite{jacob2015mobile}.
