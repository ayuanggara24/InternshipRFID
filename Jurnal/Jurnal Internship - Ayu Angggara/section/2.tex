\section{RelatedWork}
\label{Related Work}
Applications that use Radio Frequency Identification (RFID) are increasing today and are seen applicable in areas such as material flow discharge, quality assurance \cite{marcus2017system}, production control \cite{ramadan2017rfid}, \cite{tao2017advanced}, cold chain logistics tracking \cite{chen2014novel}, objects in place of find \cite{cai2014boundary}. RFID consists of two important parts of the RFID reader (combination of transceiver and antenna) and RFID tags (composed of unique numbers). The RFID tag is used to store important data from the observer while the reader is used to read the data stored in the tag. This technology has the advantage of data transfer that is contactless and able to work in every environment \cite{wang2017deep}. RFID is one of the wireless technologies that use electromagnetic signal detection as identification \cite{abas2017attendance}. Frequencies used in RFID consist of various types of frequencies such as low frequency, high frequency, ultra high frequency, and microwave \cite{srinidhi2015web}. In particular, referring to the process tracer category of the process, researchers mainly focus on manufacturing, production logistics, inventory, and supply chain. At an in-place level, RFID improves real-time data retrieval and fuses for process visibility. Propose a formal RFID-based deduction model to monitor changes in the flow of time-sensitive materials in the workshop \cite{cao2017real}. Most existing cloning detection protocols are suitable for recognisable systems, requiring knowledge of tag IDs. Such protocols to recognise IDs before detecting which IDs are related to the cloning tag and focus only on the RFID supply chain; they collect IDs from supply chain partners and detect cloned attacks when the IDs appear simultaneously in different places \cite{kang2013development}.\par
Authentication is one of the critical aspects of securing applications and systems. During the authentication process, Biometric and RFID are validation factors for verifying user identity \cite{basilio2016multifactor}, Combining user location with username and password as Multi-Factor Authentication (MFA) system to make authentication more secure \cite{Ramatsakane2017}. For secure authentication of e-voting systems using cryptographic and fingerprint IC and FTP MFA techniques \cite{oke2017developing}, Schemes using cryptography and Android enhance security, convenience, flexibility, storage efficiency and MFA performance \cite{Venukumar2016}. Using a phrase-based MFA framework to make resources on the cloud safer \cite{rehman2016framework}. By implementing several MFAs on the mobile cloud, it is possible to know the feasibility of applying the method \cite{alizadeh2014feasibility}. Hardware authentication with Fingerprint and Smartphone \cite{ba2017addressing}, as well as a combination of passwords with hybrid profiles of user behaviour with a great combination of host-based features, are also to keep user data secure \cite{uluagac2014multi}. Utilization of MFA to minimise fraud attendance data such as using face-recognising \cite{sarker2016design}, Cloud verification system that combines biometric factors and Passwords to achieve high levels of security \cite{khan2015multi}. User database on API device information that can provide that information to web applications \cite{mandyam2015leveraging}. The MFA architecture utilises Identity Federation and Single-Sign-On technologies, for the modular integration of the authentication factor \cite{shah2015multi}. Authentication security is assured because of the nature of the hash function, the combination of secret vital methods and the method of creating a one-time token \cite{zhao2015asynchronous}. Merging NFC and One Time Password (OTP) methods to improve system security and eliminate attendance cheating \cite{jacob2015mobile}. NFC-based MFA systems have better security advantages with a simple login process \cite{hufstetler2017nfc}. 
\par Based on previous research, this research will do the design to improve the security of existing attendance system. The design will use Radio Frequency Identification (RFID) card as a tool of attendance by applying Multi-Factor Authentication method. MFA is a way to authenticate users by using multiple layers of authentication programs. One example is to use OTP (One-Time Password). OTP send to each user's mobile device; this OTP is only valid for one login session or transaction and, each user has a different OTP. Thus, using the MFA and OTP method can minimise the occurrence of fraud or duplication of attendance records system.
